\section{The Piecewise Affine Model}

We construct a Piecewise affine (PWA) model $\hat{\scrS}$ defined by
$\Map{}$ to approximate the behaviors of a continuous hybrid dynamical
system.  Such a PWA model is non-deterministic, discrete-time (with a
given time step $\tau$) and captures the witnessed behavior of the
given $\scrS$ with bounded error $\epsilon$.

Although $\hat{\scrS}$ is approximate and models only the witnessed
behavior, it can guide the serach for a violating trajectory
(justify).

The behavior of $\hat{\scrS}$ are defined using the map $\Map{}$. The
PWA model partitions the state space $\ContStates$ of $\scrS$ into
overlapping regions which are (templatized) polytopes $P_{c}$ with a
fixed number of constraints $c$. For the rest of the presentation we
assume a fixed $c$ and simply denote the polytopes as $P$.  For each
poytope, the PWA model defines a map $\Map{i}: \ContStates \mapsto
\ContStates$ which predicts the next state $\x'$ after time $\tau$
given a current state $\x \in P_i$ using an affine relation $\x' = A\x + b
\pm \epsilon$. The overall PWA model can then be defined as below.

\begin{equation}
    \Map{} = \left\{
        \begin{array}{ll}
            \Map{1} = A_1\vx + \vb_1 & \vx \in P_1\\
            \ldots & \ldots\\
            \Map{i} = A_i\vx + \vb_i & \vx \in P_i\\
            \ldots & \ldots\\
            \Map{n} = A_n\vx + \vb_n & \vx \in P_n
        \end{array}
    \right.
\end{equation}

As $P_i$ can be overlapping, $\Map{}$ will be non-deterministic. The
union of all polytopes contains the state space of the system,
$\ContStates \subset \displaystyle \bigcup_i P_i$. Hence, for each
$\x\in\ContStates$, there exists atleast one polytope $P_i$ such that
$\x\in P_i$.

Given the current state of the system $\x$, the PWA model
$\hat{\scrS}$ approximates its state after time $\tau$ such that

%The discretization $\hat{\scrS}$ is qualified as $\hat{\scrS}_\epsilon$
%if it is $\delta$-approximate, \ie, the below is true under a given
%norm.
\begin{equation}
    \forall \vx \in \ContStates.
        ||\Map{}(\vx) - \simulate(\vx, \tau)|| \le \epsilon
\end{equation}

The approximated discrete system $\hat{\scrS}$ evolves by the repeated
application of the map $\Map{}$. Given a state $\vx$ of the
system at time $t$, the state at time $t + k\tau$, $\vx_k$ is
approximated by iterating $\Map{}$ over $\vx$, $k$ times.

\begin{align}
    \vx_k &= \Map{}(\vx_{k-1}) \nonumber\\
          &= \Map{}^2(\vx_{k-2}) \nonumber\\
          &= \ldots \nonumber\\
          &= \Map{}^k(\vx)
\end{align}

\begin{definition}{Trace of PWA model ($\pi$)}. The $k$-length trace
$\pi$ of $\hat{\scrS}$ is given as a ordered sequence of discrete
states $\x_0, \x_1, \ldots, \x_k$, where $\x_{i+i} = \Map{i}(\x_i)$.
\end{definition}

\begin{definition}{Counter-example of $\phi$ in $\hat{\scrS}$}. The trace $\pi$ is a
    counter-example to a given property $\phi$, if ...
\end{definition}



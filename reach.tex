Approximating a continuous plant $\scrS$ by the transition
system $\hat{\scrS}$ allows us to encode the reachability problem by a
BMC formula. As we kept the map $\Map{}$ affine, we can use
off-the-shelf SMT solvers like Z3~\cite{DeMoura+Bjorner/08/Z3} to
check it. We now formally state the problem.

\paragraph{Falsification of Time Bounded Safety Property.} Given a
polytope representing unsafe of states, $X_f: A_f\vx \le \vb_f$, a
discrete transition system $\hat{\scrS}$ defined by $\Map{}$, it's
initial set of states (also a polytope) $X_0: A_0\vx \le \vb_0$, a
finite number of steps $N$, does there exist a trajectory of the
system from an initial state $\vx_0 \in X_0$ to an unsafe set such
that $\vx_k \in X_f$, where $\vx_k = \Map{}^k(\vx)$ and $k \le N$.

\subsection{Overall Problem.}
Given the black box system's description as the forward simulator
function $\simulate$, search for a violation of $\phi$,
\begin{scriptsize}
\begin{alignat*}{1}
    &Rch^k(\hat{\scrS}, X_0, X_f)\\
    &=\exists \x_0 \in X_0.\displaystyle\bigwedge_{i=0}^{k-1}T(\x_i, \x_{i+1}) \land \x_{k} \in X_f\\
    &=\exists \x_0 \in X_0.
    \displaystyle\bigwedge_{i=0}^{k-1}
    \left(\displaystyle\bigvee_{j \in M}P_j(\x_i) \land
        \displaystyle\bigwedge_{j \in M}P_j(\x_i) \implies \x_{i+1} =
    \Map{j}(\x_i)\right) \land \x_{k} \in X_f\\
\begin{split}
    =\exists \x_0 \in X_0.\displaystyle\bigwedge_{i=0}^{k-1} \Bigg(\displaystyle\bigvee_{j \in M}C_j\x_i\le d_j \land \displaystyle\bigwedge_{j \in M}C_j\x_i\le d_j\\
    \implies \x_{i+1} = A_j\x_i + b_j \pm \epsilon \Bigg) \land \x_{k} \in X_f
\end{split}
\end{alignat*}
\end{scriptsize}
The presence of terms $C_j\x_i$ and $A_j\x_i$ make the problem
non-linear or multi-affine when both $C_j$ (similarily $A_j$) and
$\x_i$ are simulatenously unknown. We take the route of splitting the
modeling and reachability problem into two separate problems. In the
first problem $\x_i$s are finite and known and in the second one,
$C_j$ and $A_j$ are known. We still need to connect the two problems,
which we do using a model improvement/refinement procedure.


\subsection{Search for a violation.} Given a BMC formula over affine
arithmetic, an SMT solver either returns a satisfying assignment for
it or finds it unsatisfiable. The former case provide us with an
abstract violation while the latter conveys us the absence of one in
the sound abstraction. To find the corresponding concrete violation if
it exists, a concretization step is used. If a violation is not found,
we can not conclude the absence of one. Instead, we can enrich our
data set by adding more states and re-starting. Can we do anything
else?

\subsection{Concretization} An abstract path can be concretized by
using random simulations. Same as before, we use the initial abstract
state $C_0$ in the returned abstract path and sample it to get
$\scr{N}$ concrete states ($\scr{N}$ is the concretization budget).
The states are then simulated for specified time horizon $T$ using the
given simulation function $\simulate$. If a violation is triggered, we
conclude a successful falsification and stop. If not, we refine the
abstraction.

\todo{This section is not very clearly written}\\
A refinement of the abstraction can be carried
out by reducing the error in the PWA model. On the other hand, an
enrichment of the model be carried out by adding more behaviors
(states) in the data set.
Several ideas exist (discuss)
\begin{itemize}
    \item Add more $(\x, \x')$ pairs to the data set belonging to
        the traces obtained from $\simulate$ along the traces of
        the PWA model returned as counter-examples by the BMC.
    \item Decrease $\epsilon$ along relevant $(\x,\x')$. Such pairs
        can identified along the polytopes on the abstract
        counter-example. In an abstract path, either all polytopes can
        be refined or selectively refined by comparing the divergence
        between their predictions and observed numerical values from
        the trace produced by $\simulate$.
    \item Re-model to obtain a refined/enriched model.
    \item lather, rinse, repeat
\end{itemize}

\myipara{Note}The error is captured by $\epsilon$ for each state $\x$.
Hence, we can modify $\epsilon$ for each state individually. Or, we
can modify it for each polytope.

\myipara{Deleting states} $\x$ deemed useless for the property at
hand can also be deleted. However, if done ad-hoc, this takes away all
covergence guarentees in the presence of a robust violation in the
system $\scrS$. Perhaps we can determine conditions under which it
can be successfully applied.

\myipara{Increasing $\epsilon$} We can also increase
$\epsilon$ to apply certain parts of state space to be modeled more
coarsely than the rest. As above, an ad-hoc application will result in
loss of any convergence guarentees.

While re-modeling, it can happen that there does not exist a model
with the fixed number of polytopes $M$ and template $c$. In such a
case, we choose to increment $M := M + 1$ till we cna find a model. At
this point it is unclear how the number of constraints $c$ can be
manipulated alongside (discuss).

\section{Discussion}

If possible, we would like the time step $\tau$ to be dynamic
and dynamics based instead of fixing one for the entire system. This
can potentially decrease the depth $k$ required during BMC.

We would also like to work with the full set of LTL properties. How
can we check liveness properties? Can we extend this to finding
counter-examples for $\omega$ regular proerties?

Also, it should be noted that the above can be easily extended to
handle bounded and parameterized disturbances/inputs.

Even when a path is feasible, we need to use our previously defined
concretization step to verify if the found initial state leads to an
error state. To recall, this is done by simply using the $\simulate$
function and simulating $k$-times at and around the
$\delta$-neighbourhood of the feasible region, where $k$ and $\delta$
define the budget of the concretization step.

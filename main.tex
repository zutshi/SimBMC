%\documentclass{sig-alternate}
\documentclass{sig-alternate-05-2015}
\usepackage{tikz}
\usepackage{framed}
\usepackage{stmaryrd}
\usepackage{amsmath}
\usepackage{ragged2e}
\usepackage{varwidth}

\usetikzlibrary{arrows,backgrounds,positioning,fit,automata,shapes,snakes,patterns}
\usetikzlibrary{arrows,backgrounds,decorations,decorations.pathmorphing,positioning,fit,automata,shapes,snakes,patterns}
\usetikzlibrary{shapes.geometric, arrows, positioning, calc, matrix}
\tikzset{block/.style = {draw, fill=blue!20, rectangle,
                         minimum height=3em, minimum width=6em},
        sum/.style = {draw, fill=blue!20, circle, node distance=1cm},
        input/.style = {coordinate},
        output/.style = {coordinate},
        pinstyle/.style = {pin edge={to-,thin,black}}
}

\usepackage{balance}
\usepackage{cite}

\makeatletter
\newif\if@restonecol
\makeatother
\let\algorithm\relax
\let\endalgorithm\relax

\renewcommand{\baselinestretch}{0.97}

\usepackage{times}
%\usepackage{graphicx,enumerate}
\usepackage{graphicx}
\usepackage{enumitem}
%\usepackage{enumerate}
\usepackage{subfig}
%\usepackage{caption}
%\usepackage{subcaption}
\usepackage{amsmath,amssymb,amsfonts}
\usepackage{fancyhdr}
\usepackage{balance}
\usepackage{cite}
\usepackage{color}
\usepackage{wrapfig}
\usepackage[ruled,vlined,linesnumbered]{algorithm2e}
%\usepackage{booktabs}
\usepackage{booktabs,siunitx}
\usepackage{xcolor,colortbl}
\usepackage{times}
\usepackage{microtype}
\usepackage{url}
\usepackage{listings}
\usepackage{multirow}
%\usepackage{tablefootnote}

% \usepackage{hyperref}
% \hypersetup{
%     colorlinks=false,
%     pdfborder={0 0 0},
% }

%%\usepackage{abstract}

\makeatletter
\def\url@mystyle{%
  %\@ifundefined{selectfont}{\def\UrlFont{\sf}}{\def\UrlFont{\small\ttfamily}}}
  \@ifundefined{selectfont}{\def\UrlFont{\mathtt}}{\def\UrlFont{}}}
  %\@ifundefined{selectfont}{\def\UrlFont{\sf}}{\def\UrlFont{\small}}}
\makeatother
%\urlstyle{mystyle}
\urlstyle{rm}


\definecolor{mygreen}{rgb}{0,0.6,0}
\definecolor{mygray}{rgb}{0.5,0.5,0.5}
\definecolor{mymauve}{rgb}{0.58,0,0.82}

\newtheorem{example}{Example}[section]
\newtheorem{definition}{Definition}[section]
\newtheorem{assumption}{Assumption}[section]
\newtheorem{lemma}{Lemma}[section]
\newtheorem{theorem}{Theorem}[section]

%TODO: Fix line 738 of sig-alternate-05-2015.cls for the copyright
%notice to appear
\DeclareCaptionType{copyrightbox}


% Generic
%\DeclareMathAlphabet{\mathpzc}{OT1}{pzc}{m}{it}

% English
\newcommand{\ie}{{i.e.}\xspace}
\newcommand{\Ie}{{I.e.}\xspace}
\newcommand{\eg}{{e.g.}\xspace}
\newcommand{\etc}{{etc.}\xspace}
\newcommand{\viz}{{viz.\xspace}}
\newcommand{\etal}{{et al.}\xspace}

\renewcommand\vec[1]{\mathbf{#1}}
% Generic refs
\newcommand{\lemref}[1]{Lemma~\ref{lem:#1}}
\newcommand{\secref}[1]{Sec.~\ref{sec:#1}}
\newcommand{\figref}[1]{Fig.~\ref{fig:#1}}
\newcommand{\exref}[1]{Example~\ref{ex:#1}}
\newcommand{\thmref}[1]{Theorem~\ref{thm:#1}}
\newcommand{\tabref}[1]{Table~\ref{tab:#1}}
\newcommand{\defref}[1]{Definition~\ref{def:#1}}
\newcommand{\algoref}[1]{Algorithm~\ref{alg:#1}}
\newcommand{\lstref}[1]{Listing~\ref{lst:#1}}
%\newcommand{\eqref}[1]{Equation~\ref{eq:#1}}

% Comments, Reviewing, Formatting
\newcommand{\ignore}[1]{}
\newcommand\todo[1]{[[\textcolor{red}{\textsf{#1}}]]}
% General Math
\newcommand{\setof}[1]{\ensuremath{\{#1\}}}
\newcommand\tupleof[1]{\ensuremath\left\langle #1 \right \rangle}
\newcommand{\sublist}[2]{\ensuremath{#1_{1},\ldots,#1_{#2}}}
\newcommand{\suplist}[2]{\ensuremath{#1^{1},\ldots,#1^{#2}}}
\newcommand \card[1] {\left| #1 \right|}
\newcommand \floor[1] {\left\lfloor #1 \right\rfloor}
\newcommand \ceil[1] {\left\lceil #1 \right\rceil}

% Complexity
\newcommand{\npcomplete}{\textsc{Np}-\textsc{complete}}
\newcommand{\nphard}{\textsc{Np}-\textsc{Hard}}
\newcommand{\nlogspace}{\textsc{NLogspace}\xspace}
\newcommand{\pspace}{\textsc{PSpace}\xspace}
\newcommand{\expspace}{\textsc{ExpSpace}\xspace}

\newcommand{\mcl}[1]{\multicolumn{1}{l}{#1}}
\newcommand{\mcc}[1]{\multicolumn{1}{c}{#1}}

\newcommand{\mypara}[1]{\vspace{0.6em} \noindent{\bf #1.}}
\newcommand{\myipara}[1]{\vspace{0.6em} \noindent{\em #1.}}
\newcommand{\myiparaCompact}[1]{ {\em #1}\xspace}

% Sets
\newcommand{\Reals}{\ensuremath{\mathbb{R}}}
\newcommand{\reals}{\Reals}
\newcommand{\Nats}{\ensuremath{\mathbb{N}}}
\newcommand{\Integers}{\ensuremath{\mathbb{Z}}}
\newcommand{\Rationals}{\ensuremath{\mathbb{Q}}}

%%\newcommand{\mathsf}[1]{\mbox{\textsc{#1}}}
\newcommand{\mathsc}[1]{\mbox{\sc #1}}
\newcommand\scr[1]{\ensuremath\mathcal{#1}}
\newcommand\Gradient{\ensuremath\nabla}
\newcommand\pdiff[2]{\partial_{#1}{#2}}
\newcommand\jth[2]{ #1^{(#2)} }

% Logic
% \newcommand{\land}{\ensuremath\wedge}
% \newcommand{\lor}{\ensuremath\vee}

% Software
\newcommand\MATLAB{Matlab\texttrademark\;}
\newcommand\SIMULINK{Simulink\texttrademark\;}
\newcommand\STATEFLOW{Stateflow\texttrademark\;}
\newcommand\EMCODER{Embedded Coder\texttrademark\;}
\newcommand\fmincon{\texttt{fmincon}\xspace}


\newcommand\vs{\mathbf{s}}
\newcommand\vw{\mathbf{w}}
\newcommand\vx{\mathbf{x}}
\newcommand\vy{\mathbf{y}}
\newcommand\vz{\mathbf{z}}
\newcommand\vu{\mathbf{u}}
\newcommand\vj{\mathbf{j}}

\newcounter{sarrow}
\newcommand\xrsquigarrow[1]{%
\stepcounter{sarrow}%
\begin{tikzpicture}[decoration=snake]
\node (\thesarrow) {\strut#1};
\draw[->,decorate] (\thesarrow.south west) -- (\thesarrow.south east);
\end{tikzpicture}%
}

% Units
\newcommand{\degree}{^{\circ}}
\newcommand{\degreeC}{^{\circ}{\rm C}}
\newcommand{\degreeF}{^{\circ}{\rm F}}
% misc
\newcommand\ii{i+1}
\newcommand\denotation[1]{ \left\llbracket #1 \right\rrbracket}

% RRT
\newcommand{\RRT}{\mathcal{RRT}}
\newcommand{\RRTgoal}{\x_{goal}}
\newcommand{\RRTinit}{\x_{init}}
\newcommand{\RRTv}[1]{v_{#1}}
\newcommand{\RRTe}{e}
\newcommand{\RRTV}{V}
\newcommand{\RRTE}{E}
\newcommand{\RRTsample}{x_{sample}}
\newcommand{\RRTnear}{x_{near}}
\newcommand{\RRTtoEx}{x_e}
\newcommand{\RRTxNew}{x_{new}}
\newcommand{\RRTuNew}{u_{new}}
\newcommand{\RRTdedge}[3]{(#1 \xrightarrow{#3} #2)}

% Graph
\newcommand{\Graph}{\mathbf{G}}
\newcommand{\graph}{\mathbf{G}}
\newcommand{\vertex}[1]{v_{#1}}
%\newcommand{\vertex}[1]{#1}
\newcommand{\dedge}[2]{e_{(\vertex{#1}\rightarrow\vertex{#2})}}
\newcommand{\edge}{e}
\newcommand{\vertexSet}{V}
\newcommand{\edgeSet}{E}
%\newcommand{\vertexSetD}{V^\delta}
%\newcommand{\edgeSetD}{E^\delta}
%\newcommand{\graphVE}[1]{\mathbf{G_{#1}(\vertexSet,\edgeSet)}}
%\newcommand{\graphD}[1]{\mathbf{G^\delta_{#1}(\vertexSet^\delta,\edgeSet^\delta)}}
\newcommand{\kPaths}{k\_paths}
\newcommand{\trajToNode}{\Gamma}
\newcommand{\Path}{\mathbf{P}}
\newcommand{\weight}{\mathbf{W}}

% Hybrid Automata
\newcommand{\HA}{\ensuremath{\mathcal{A}}}

\newcommand{\System}{\ensuremath{\mathcal{S}}}
\newcommand{\Inputs}{\ensuremath{\mathcal{U}}}
\newcommand{\Flow}{\ensuremath{\mathcal{F}}}
\newcommand{\Modes}{\ensuremath{\mathcal{Q}}}
\newcommand{\ContStates}{\ensuremath{X}}
\newcommand{\Inv}{\ensuremath{\mathcal{I}}}
%\newcommand{\Transitions}{\ensuremath{\Delta}}
\newcommand{\Transitions}{\scr{T}}
\newcommand{\HybridStates}{\ensuremath{\mathcal{X}}}
\newcommand{\HybridStateSet}{\ensuremath{X}}
\newcommand{\ResetMap}{\mathcal{R}}
\newcommand{\Guards}{\mathcal{G}}
\newcommand{\Init}{\mathcal{X}_{0}}
\newcommand{\Err}{\ContStates_{f}}
\newcommand{\initmode}{m_{init}}
\newcommand{\cInit}{\ensuremath{X}_0}
\newcommand{\Unsafe}{\mathcal{X}_f}
%\newcommand{\reachSet}[2]{R^{#1}_{\HA}({#2})}
%\newcommand{\Hflow}{\mathcal{H}_{\HA}}
\newcommand{\Hflow}{\mathcal{H}}
\newcommand{\reachSet}{R}

% Trajectories
\newcommand{\discTraj}{q_h}
\newcommand{\hybridTraj}{\tau_h}
\newcommand{\traj}{\pi}
\newcommand{\trajSeg}[2]{\pi^{\mode_{#1}}_{\tau_{#2}}}
\newcommand{\SegTraj}{\mathbf{S}_\pi}
\newcommand{\dwell}{\tau}
\newcommand{\tran}{\delta}
\newcommand{\tbegin}{b}
\newcommand{\tend}{e}
%\newcommand{\graph}{G}
\newcommand{\trajStore}{TS}
\newcommand{\trajSet}{TS}
\newcommand{\candTraj}{CT}
\newcommand{\candTrajSet}{CTS}
\newcommand\Cost{\mathsc{Cost}}


% State variables
\newcommand{\x}{\mathbf{x}}
\newcommand{\y}{\mathbf{y}}
\newcommand{\z}{\mathbf{z}}
\newcommand{\w}{\mathbf{w}}
%\newcommand{\vec}[1]{\mathbf{#1}}
\newcommand{\dvx}{\mathbf{\dot{x}}}
\newcommand{\inp}{\mathbf{u}}
\newcommand{\mode}{\ensuremath{q}}
\newcommand{\dx}{\ensuremath{\dot{x}}}

% Metrics
\newcommand{\lmetric}[2]{{\Vert #2 \Vert}_#1}

% ODE
\newcommand{\odesolution}{\Phi}
\newcommand{\matA}{\mathbf{A}}
\newcommand{\matB}{\mathbf{B}}
\newcommand{\mata}{\mathbf{a}}


\newcommand \maximize{\mathbf{max.}\ }
\newcommand \minimize{\mathbf{min.}\ }

\newcommand \numsolve{\mathtt{NumericSolver}}
\newcommand\Flowmap{\mathsc{Flow}}
%%\newcommand\myipara[1]{\par\noindent\textit{#1:}}

% Arrows
%\newcommand{\contArrow}[2]{\underset{#2}{\overset{#1}{\leadsto}}}
\newcommand{\contArrow}[1]{\leadsto_{#1}}
\newcommand{\jumpArrow}[1]{\xrightarrow{#1}}


\newcommand\dist{\mathsf{d}}
\newcommand\src{\mathsf{src}}
\newcommand\dest{\mathsf{dest}}
\newcommand\timeElapse{\mathcal{T}}
\newcommand\simulate{\mathsc{sim}}
%%\newcommand\sim{\mathsc{sim}}
\newcommand\cost{\mathsc{cost}}
\newcommand\reach[1]{\xrightarrow{#1}}
\newcommand\areach[1]{\overset{#1}{\rightsquigarrow}}

\newcommand\intr{\mathsf{interior}}
\newcommand\assign{:= }
\newcommand\worklist{\mathsf{workList}}
\newcommand\exploredCells{\mathsf{V}}
\newcommand\exploredEdges{\mathsf{E}}
\newcommand\unsafeCells{\mathsf{V}_u}
\newcommand\initialCells{\mathsf{V}_0}
\newcommand\prb{\mathbb{P}}
\newcommand{\abstracteps}{\epsilon}
\newcommand{\refinedeps}{\delta}
\newcommand{\absgraph}{\scr{H}_{\epsilon}(\Delta)}
\newcommand{\refgraph}{\scr{H}_{\delta}(\Delta)}
\newcommand{\abscells}{{\scr{C}}}
\newcommand{\refcells}{{\scr{D}}}






%%%%%%%% from sriram sty, rearrange!

\def\mathsc#1{\mbox{\sc #1}}

\newcommand\trp[1] {#1^{\scriptscriptstyle T}}
\newcommand\scrS{\mathcal{S}}
\newcommand\scrT{\mathcal{T}}
\newcommand\scrF{\mathcal{F}}
\newcommand\scrG{\mathcal{G}}
\newcommand\scrH{\mathcal{H}}
\newcommand \ints {\ensuremath \mathbb{Z}}
%%\newcommand \extreals {\ensuremath \mathcal{R}^{+}}
\newcommand \lin[1]{\mathit{Lin}(#1)}
\newcommand \conic[1]{\mathit{Cone}(#1)}
\newcommand \conv[1]{\mathit{Convex}(#1)}
\newcommand \false {\mathit{false}}
\newcommand \true  {\mathit{true}}
%\newcommand\pre{\preceq}
\newcommand \pres{\mathbf{pres}}
\newcommand \F {\mathfrak{F}}

\newcommand \abstractF {\F_A}
\newcommand \abstractDomain {\Sigma_A}
\newcommand \templateDomain {\Sigma_T}
\newcommand \dropped {\mathsc{x}}
\newcommand \ctop {\vec{c}_\top}
\newcommand \cbot {\vec{c}_\bot}
\newcommand \concT {\gamma_T}
\newcommand \abstractorder {\leq_A}
\newcommand \abstractLeq{\sqsubseteq}
\newcommand \abstractor {\sqcup}
\newcommand \abstractand {\sqcap}
\newcommand \bigabstractor {\bigsqcup}
\newcommand \setdef[2] { \left\{ #1 \mid #2 \right\}}

\newcommand \CH {\mathcal{CH}}


%%\newcommand \mathsc[1]{\mbox{\textsc{#1}}}
%%transpose



\newcommand \T {\mathcal{T}}
\newcommand \post {\ensuremath\mathit{post}}
\newcommand \dpost {\widehat{\mathit{post}}}
\newcommand \homg[1] {\mathsc{hom}(#1)}
\newcommand \cons{\mathsc{cons}}
\newcommand \widen {\nabla}
\newcommand \dual[1]{\widehat{#1}}
\newcommand \narrow {\/ \bigtriangleup \/ }
\newcommand \refine {\partial}
\newcommand \deta{\pi}

\newcommand \latcap {\sqcap}
\newcommand \latcup {\sqcup}
\newcommand \guard {\xi}
\newcommand \templ{\gamma}
\newcommand \sizeof[1] { |#1| }
\newcommand \uset[2]{\underset{#1}{\underbrace{#2}}}
\newcommand \ith[2]{ {#1}^{(#2)}}
\newcommand \C {\ensuremath \mathcal{C}}
\newcommand \complex {\ensuremath \mathcal{C}}
\newcommand \grb {Gr\"obner\xspace}
\newcommand \hi[1]{}
\newcommand \hihenny[1]{}
\newcommand \newhenny[1]{#1}
\newcommand\markmargin[2]
{[\marginpar[\hfill \mbox{#1}$\rightarrow$]{$\leftarrow$\mbox{#1}} 
{\sf #2]}}

\newcommand\highl[1]{\psframebox[fillstyle=solid,fillcolor=lightgray,linewidth=0pt]{\textbf{#1}}}


\newcommand \locs{\mathbf{L}}
\newcommand\en{\mathbf{en}}




\newcommand{\chapterheading}[1]
{\vfill  
\hfill \fbox{\begin{minipage}{5.5in}
\textsl{#1} \end{minipage} } \newpage}

\newcommand \Z{ \mathbf{Z}}

\newcommand \mand {\ \mathit{and}\ }
\newcommand \matb{\vec{b}}
\newcommand \matl{\vec{\lambda}}
\newcommand \vg {\vec{g}}
\newcommand \va{\vec{a}}
\newcommand\ve{\vec{e}}
\newcommand \vc {\vec{c}}
\newcommand \vf {\vec{f}}
\newcommand \vh {\vec{h}}
\newcommand \vv{\vec{v}}
\newcommand \vzero {\vec{0}}
\newcommand \vq {\vec{q}}
\newcommand \rank {\mathit{rn}}
\newcommand\vbeta{\vec{\rho}}
\newcommand \Petri {\ensuremath \mathcal{P}}
\newcommand \maxrank {\mathit{maxrank}}
\newcommand \lists{\mathit{list}}
\newcommand \nullsp{\mathit{null}} 
\newcommand \vl[1] {\vec{\lambda_{#1}}}
\newcommand\vlam{\vec{\lambda}}
\newcommand \st{\mathbf{s.t.}\ }


\newcommand \D {\mathit{D}}
\newcommand \I {\mathit{I}}
\newcommand \Loc{\mathit{Loc}}
\newcommand \lie{\mathcal{L}}
\newcommand \grad{\nabla}
\newcommand \cn {\mathsf{Cn}}

\newcommand \diff[2] {\ensuremath \frac{d #1}{d #2}}




\renewcommand\paragraph[1] {\smallskip\par\noindent\textbf{#1}\ \ }
\newcommand \cplus {\uplus}

\newcommand\relop{\mathop{\bowtie}}
\newcommand \vt{\vec{t}}
\newcommand \vd{\vec{d}}
\newcommand\e{\mathsf{e}}
\newcommand\yl{\mathsf{l}}
\newcommand\yu{\mathsf{u}}
\newcommand\yb{\mathsf{b}}


\newcommand\init{\mathsf{init}}

\newcommand\ol[1]{\overline{#1}}
\newcommand\state[1]{\vec{{\texttt{#1}}}}

\newcommand\scrP{\mathcal{P}}
\newcommand\scrX{\mathcal{X}}
\newcommand\scrC{\mathcal{C}}
\newcommand\interVal[1]{[\underline{#1}, \overline{#1}]}
\newcommand\HIDE[1]{}


\newcommand{\HErr}{\mathcal{X}_{f}}
\newcommand\Eq{\mathsf{Eq}}
\newcommand\Ineq{\mathsf{Ineq}}
%\newcommand \setof[1] { \left\{ #1 \right \}}
%\newcommand \vx {\vec{x}}
%\newcommand \vy {\vec{y}}
%\newcommand \vz {\vec{z}}
%\newcommand \vu {\vec{u}}
\newcommand \vb{\vec{b}}
%\newcommand \pdiff[2] { \ensuremath \frac{\partial #1}{\partial #2}}
%\newcommand \reach[1]{\mathsf{Reach}(#1)}


\newcommand{\vertexSetD}{V^\delta}
\newcommand{\edgeSetD}{E^\delta}
\newcommand{\graphVE}[1]{\mathbf{G_{#1}(\vertexSet,\edgeSet)}}
\newcommand{\graphD}[1]{\mathbf{G^\delta_{#1}(\vertexSet^\delta,\edgeSet^\delta)}}



% plant and controller

\newcommand\pgm{\rho}
%\newcommand\outputs{Y}
%\newcommand\locs{L}
%\newcommand\sloc{l_i}
%\newcommand\eloc{l_o}
%\newcommand\op{op}
%\newcommand\expr{E}


\newcommand\symMem{\mu}
\newcommand\CFG{\Pi}
\newcommand\pLocs{L}
\newcommand\pEdges{E}
\newcommand\pLabels{\Phi}
\newcommand\pEdge[1]{edge(#1)}
\newcommand\pVars{\mathcal{V}}
\newcommand\pVar{v}
\newcommand\pTransRel{\rho}
\newcommand\pLoc{l}
\newcommand\pLocI{l_0}
\newcommand\pLocF{l_f}
\newcommand\pVarI{V_0}
\newcommand\pPaths{\mathcal{P}}
\newcommand\pPath{p}
\newcommand\pPathCons{\kappa}
\newcommand\pCons{\xi}
\newcommand\pSymMem{\sigma}
\newcommand{\domain}{\mathcal{D}}
\newcommand{\pre}{\mathit{pre}}
\newcommand{\update}{\mathit{update}}
%TODO: redef?? fixit!
%\newcommand{\C}{\mathtt{C}}

\newcommand{\sampletime}{{\tau_{s}}}

\newcommand\pgmF{\rho}
\newcommand\pgmA{\hat{\rho}}
\newcommand\plantStates{x}
\newcommand\abstractPlantStates{X}
\newcommand\controllerOutputs{u}
\newcommand\controllerStates{s}
\newcommand\PathCons{\kappa}
\newcommand\constraints{\psi}
%TODO: redinfe path...conflicting with prev def
%\newcommand\Path{\mathcal{P}}
\newcommand\absState{\mathcal{A}}

\newcommand\ts{\tau_s}

\newcommand\Cix{C_i^x}
\newcommand\Ciy{C_i^y}
\newcommand\Ciix{C_{i+1}^x}
\newcommand\Ciiy{C_{i+1}^y}
% \newcommand\si{s_i}    %conflicts with package siuntix
\newcommand\sii{s_{i+1}}
\newcommand\ui{u_i}
\newcommand\uii{u_{i+1}}
\newcommand\sip[1]{s_{i}^{p_{#1}}}
\newcommand\uip[1]{u_{i}^{p_{#1}}}
\newcommand\siip[1]{s_{i+1}^{p_{#1}}}
\newcommand\uiip[1]{u_{i+1}^{p_{#1}}}
\newcommand\kpi{\kappa^{p_i}}


\renewcommand{\Si}{S_i}
%\newcommand\Si{S_i}
\newcommand\Sii{S_{i+1}}
\newcommand\Ui{U_i}
\newcommand\Uii{U_{i+1}}
\newcommand\Sip[1]{S_{i}^{p_{#1}}}
\newcommand\Uip[1]{U_{i}^{p_{#1}}}
\newcommand\Siip[1]{S_{i+1}^{p_{#1}}}
\newcommand\Uiip[1]{U_{i+1}^{p_{#1}}}

\newcommand \ctrl{\mathsc{ctrl}}
\newcommand\quant{\mathsc{Quant}}
\newcommand\sample{\mathsc{Sample}}

\newcommand\TrajOpt{TrajOpt\;}

\newcommand{\cellequivalence}{\equiv_{\scr{C}}}
\newcommand{\vk}{\mathbf{k}}


%\clubpenalty=10000
%\widowpenalty = 10000

\title{Bounded Model Checking of Hybrid Dynamical Systems using Numerical Simulations}

\numberofauthors{3}
\author{
\alignauthor
Aditya Zutshi\\
\affaddr{\small{University of Colorado, Boulder}}\\
\email{\small{aditya.zutshi@colorado.edu}}\\
    \and
\alignauthor
Sergio Mover\\
\affaddr{\small{University of Colorado, Boulder}}\\
\email{\small{sergio.mover@colorado.edu}}
  \and
\alignauthor
Sriram Sankaranarayanan\\
\affaddr{\small{University of Colorado, Boulder}}\\
\email{\small{srirams@colorado.edu}}
}

\date{\today}

\begin{document}

%\CopyrightYear{2016}
%\setcopyright{acmcopyright}
%\conferenceinfo{HSCC'16,}{April 12-14, 2016, Vienna, Austria}
%\isbn{978-1-4503-3955-1/16/04}\acmPrice{\$15.00}
%\doi{http://dx.doi.org/10.1145/2883817.2883819}

\maketitle

\begin{abstract}
    In this work we address the problem of findfing behaviors violating a
given time bounded LTL property in hybrid dynamical systems. We
approach the problem in two distinct steps. First, we approximate the
behavior of the underlying system using a piecewise discrete-time
affine model. This model incomplete in nature and is computed with
respect to the given property.  We then encode the falsification
search as a bounded model checking query and use an SMT solver to find
a counter example. If found, we check if violation is also satisfied
by the given numerical simulation engine. If it is not, we refine our
models and repeat.

\end{abstract}

%
% The code below should be generated by the tool at
% http://dl.acm.org/ccs.cfm
% Please copy and paste the code instead of the example below.
%
\begin{CCSXML}
<ccs2012>
<concept>
<concept_id>10010147.10010341.10010342.10010344</concept_id>
<concept_desc>Computing methodologies~Model verification and validation</concept_desc>
<concept_significance>500</concept_significance>
</concept>
<concept>
<concept_id>10011007.10011074.10011099.10011692</concept_id>
<concept_desc>Software and its engineering~Formal software verification</concept_desc>
<concept_significance>300</concept_significance>
</concept>
</ccs2012>
\end{CCSXML}

\ccsdesc[500]{Computing methodologies~Model verification and validation}
\ccsdesc[300]{Software and its engineering~Formal software verification}

%
%  Use this command to print the description
%
%\printccsdesc

%\keywords{Reachability; Hybrid Systems; Falsification}

\newcommand{\RA}[1]{R^{#1}}
\newcommand{\Map}[1]{f_{#1}}

%%%%%%%%%%%%%%%%%%%%%%%%%%%%%%
\section{Introduction}
\label{sec:intro}
We assume that the continuous-time hybrid dynamical system under test
can be numerically simulated using a given $\simulate$ function. SUch
a function is a state based deterministic numerical simulator of the
form $\x'=\simulate(t,\x)$, where a new state $\x'$ reachable after
time $t$ from $\x$ can be computed numerically whithin `acceptable'
precision. Given a time bounded LTL property $\phi$, we now present a
best effort search to find a counter-example $\pi$ to $\phi$ which can
be reproduced using the given $\simulate$.

What and Why?
\begin{itemize}
    \item Verification/falsification approaches use model
        based on symbolic dynamics. SECAM finds falsifications using
        only simulations. Can we find an approach which is of
        intermediate complexity by recovering some model of the system
        from observed simulations? We can use local PWA modeling.
\end{itemize}

How?
\begin{itemize}
    \item Generate data: relations
    \item Build local PWA models of the system by discretizing space
          and time.
    \item Use an SMT solver to check time bounded
          properties (Bounded model checking).
    \item Refine
\end{itemize}

What deficiencies are we addressing here of the current
appraoches (SECAM, S-Taliro, others)?

\subsection{Building a Model} We would like to build a model of the
system that is amenable to falsification (or verification in the case
of white-box systems). A naive solution would be to describe all the
behaviors of the system using a PWA model, where each region on the
state space is associated with a set of affine dynamics. Such a
discretization of the continuous hybrid dynamical system is amenable
to model checking and has been explored in the past in several
works~\cite{}. The problem lies in the scalability. Depending upon the
dynamics of the system, an intractable number of paritions might be
necessary to get accurate enough PWA model. Moreover, identifying such
paritions automatically is a hard problem~\cite{} (check nimit's paper
for citations).

\subsection{Finding a counterexample} Once we have the pwa model we
use an SMT solver to check the property $\phi$ for bounded time.

\subsection{Refinement} Once a counterexample for $\phi$ against the
pwa model is found, it needs to be checked if the numerical simulator
satisfies it. This is due to the discrepencies between the approximate
PWA model and numerical simulator. A confirmed violation at this stage
successfully terminates the search for the violation. Otherwise, a
refinement of the PWA model is necessary.
Refinement ideas?

%%%%%%%%%%%%%%%%%%%%%%%%%%%%%%

%%%%%%%%%%%%%%%%%%%%%%%%%%%%%%
\subsection{Motivation}
\label{sec:mot}
\input{motivation.tex}
%%%%%%%%%%%%%%%%%%%%%%%%%%%%%%

%%%%%%%%%%%%%%%%%%%%%%%%%%%%%%
\section{Related Work}
\label{sec:rel}
\input{related.tex}
%%%%%%%%%%%%%%%%%%%%%%%%%%%%%%

%%%%%%%%%%%%%%%%%%%%%%%%%%%%%%
\section{Prelims}
\label{sec:prelims}
\section{The Piecewise Affine Model}

We construct a Piecewise affine (PWA) model $\hat{\scrS}$ defined by
$\Map{}$ to approximate the behaviors of a continuous hybrid dynamical
system.  Such a PWA model is non-deterministic, discrete-time (with a
given time step $\tau$) and captures the witnessed behavior of the
given $\scrS$ with bounded error $\epsilon$.

Although $\hat{\scrS}$ is approximate and models only the witnessed
behavior, it can guide the serach for a violating trajectory
(justify).

The behavior of $\hat{\scrS}$ are defined using the map $\Map{}$. The
PWA model partitions the state space $\ContStates$ of $\scrS$ into
overlapping regions which are (templatized) polytopes $P_{c}$ with a
fixed number of constraints $c$. For the rest of the presentation we
assume a fixed $c$ and simply denote the polytopes as $P$.  For each
poytope, the PWA model defines a map $\Map{i}: \ContStates \mapsto
\ContStates$ which predicts the next state $\x'$ after time $\tau$
given a current state $\x \in P_i$ using an affine relation $\x' = A\x + b
\pm \epsilon$. The overall PWA model can then be defined as below.

\begin{equation}
    \Map{} = \left\{
        \begin{array}{ll}
            \Map{1} = A_1\vx + \vb_1 & \vx \in P_1\\
            \ldots & \ldots\\
            \Map{i} = A_i\vx + \vb_i & \vx \in P_i\\
            \ldots & \ldots\\
            \Map{n} = A_n\vx + \vb_n & \vx \in P_n
        \end{array}
    \right.
\end{equation}

As $P_i$ can be overlapping, $\Map{}$ will be non-deterministic. The
union of all polytopes contains the state space of the system,
$\ContStates \subset \displaystyle \bigcup_i P_i$. Hence, for each
$\x\in\ContStates$, there exists atleast one polytope $P_i$ such that
$\x\in P_i$.

Given the current state of the system $\x$, the PWA model
$\hat{\scrS}$ approximates its state after time $\tau$ such that

%The discretization $\hat{\scrS}$ is qualified as $\hat{\scrS}_\epsilon$
%if it is $\delta$-approximate, \ie, the below is true under a given
%norm.
\begin{equation}
    \forall \vx \in \ContStates.
        ||\Map{}(\vx) - \simulate(\vx, \tau)|| \le \epsilon
\end{equation}

The approximated discrete system $\hat{\scrS}$ evolves by the repeated
application of the map $\Map{}$. Given a state $\vx$ of the
system at time $t$, the state at time $t + k\tau$, $\vx_k$ is
approximated by iterating $\Map{}$ over $\vx$, $k$ times.

\begin{align}
    \vx_k &= \Map{}(\vx_{k-1}) \nonumber\\
          &= \Map{}^2(\vx_{k-2}) \nonumber\\
          &= \ldots \nonumber\\
          &= \Map{}^k(\vx)
\end{align}

\begin{definition}{Trace of PWA model ($\pi$)}. The $k$-length trace
$\pi$ of $\hat{\scrS}$ is given as a ordered sequence of discrete
states $\x_0, \x_1, \ldots, \x_k$, where $\x_{i+i} = \Map{i}(\x_i)$.
\end{definition}

\begin{definition}{Counter-example of $\phi$ in $\hat{\scrS}$}. The trace $\pi$ is a
    counter-example to a given property $\phi$, if ...
\end{definition}



%%%%%%%%%%%%%%%%%%%%%%%%%%%%%%

%%%%%%%%%%%%%%%%%%%%%%%%%%%%%%
\section{Constructing the Model}
\label{sec:modeling}
\section{Generating a PWA Model}

\myipara{Problem Statement.}
Given a set of $N$ states, $D_N = {\x_0,\dots,\x_n}$, error bound
$\epsilon$, find $M$ polytopes $P_{c}$, such that the below holds.

\[\forall\x\in\D_N.\exists P_j \in P.P_j(\x)\land||\Map{i}(\x) - \simulate(\tau,\x)||\le\epsilon\]

This can be stated as the below set of constraints.
\[\displaystyle\bigwedge_{\x\in D_N} \left(
        \displaystyle\bigvee_{j\in M} P_j(\x)
        \land
        \displaystyle\bigvee_{j\in M} P_j(\x)
\implies ||\Map{j}(\x) - \simulate(\tau,\x)||\le\epsilon\right)\]

In other words, we want to construct a PWA model $\hat{\scrS}$ which
predicts the next state of the system $\scrS$ within an error bounds
of $\epsilon$. Note that this only requires that for every $\x$, there
should exist atleast one polytope which predicts the next state within
the given error bounds. Hence, it is possible that other overlapping
polytopes also contain $\x$ but their asociated maps do not provide a
`good' prediction of the next state.
Such a definition might be provide for richer models with the same
number of polytopes, but it needs to be seen/proven.

The above requirement can be strengthened by searching for the PWA
model where if a polytope contains $\x$, then its associated map must
provide a `good' prediction of the next state.

\[\forall\x\in\D_N.\forall P_j \in P.P_j(\x) \implies ||\Map{i}(\x) - \simulate(\tau,\x)||\le\epsilon\]

Re-writing the above without the quantifiers, we get the below.
\[\displaystyle\bigwedge_{\x\in D_N} \left(
        \displaystyle\bigvee_{j\in M} P_j(\x)
        \land
        \displaystyle\bigwedge_{j\in M} P_j(\x)
\implies ||\Map{j}(\x) - \simulate(\tau,\x)||\le\epsilon\right)\]

As a reminder, the constraint $P_j(\x)$ can be expanded as a set of
linear constraints $C_j(\x) \le d_j$, where $C_j$ is a matrix of size
$c\times|\x|$, and $d_j$ is a vector of length $c$, where $|\x|$
represents the elements in the state vector or its dimensions.
Similarily, $\Map{j}(\x)$ represents affine map $\Map{j}(\x) =
A_j \x + b_j \pm \epsilon$, where the square matrix $A_j$ is of size
$|\x|$ and the vectors $b_j$ and $\epsilon$ are of length $|\x|$. As
the states $\x$ are pre-determined, $\simulate(\tau,\x)$ can be
computed to be a constant and the overall modeling problem is
reduced to set of affine constraints with disjunctions. Such a problem
can be solved using an SMT solver to obtain $C_j, d_j$ specifying the
$P_j$ and their associated $\Map{j}$, specified by $A_j, b_j$.

\subsection{Discussion}
It is not clear if we should also enforce the condition, that the
entire state space (region of interest) is contained in the union of
the polytopes $\displaystyle\bigcup_{j\in M} P_j \supseteq \ContStates$.
Instead, for now, we only talk about modeling all the states in the
data set.

Another important aspect of the formulation is letting the polytopes
overlap. This can be easily strenthened to uniqeness, so that every
state $\x \in D_N$ is contained by a unique $P_j$. Hence, the
resulting PWA model becomes deterministic. Although it potentially
simplify the reachability problem, it will also weaken the power of
the PWA model (more polytopes might be required to fit the same data
set).

As of now we have fixed the $\epsilon$ for the entire PWA model. This
will change when we introduce refinement.

\subsection{PWA Model as a Transition System}
The PWA model $\hat{\scrS}$ is naturally a transition system, of the
form $P_j(\x) \implies \x' := A_j \x + b_j \pm \epsilon$. Let
$T(\x,\x')$ represent the transition relation such that $\x' =
\Map{j}(\x)$.

%%%%%%%%%%%%%%%%%%%%%%%%%%%%%%

%%%%%%%%%%%%%%%%%%%%%%%%%%%%%%
\section{Reachability Analyses}
\label{sec:reach}
Approximating a continuous plant $\scrS$ by the transition
system $\hat{\scrS}$ allows us to encode the reachability problem by a
BMC formula. As we kept the map $\Map{}$ affine, we can use
off-the-shelf SMT solvers like Z3~\cite{DeMoura+Bjorner/08/Z3} to
check it. We now formally state the problem.

\paragraph{Falsification of Time Bounded Safety Property.} Given a
polytope representing unsafe of states, $X_f: A_f\vx \le \vb_f$, a
discrete transition system $\hat{\scrS}$ defined by $\Map{}$, it's
initial set of states (also a polytope) $X_0: A_0\vx \le \vb_0$, a
finite number of steps $N$, does there exist a trajectory of the
system from an initial state $\vx_0 \in X_0$ to an unsafe set such
that $\vx_k \in X_f$, where $\vx_k = \Map{}^k(\vx)$ and $k \le N$.

\subsection{Overall Problem.}
Given the black box system's description as the forward simulator
function $\simulate$, search for a violation of $\phi$,
\begin{scriptsize}
\begin{alignat*}{1}
    &Rch^k(\hat{\scrS}, X_0, X_f)\\
    &=\exists \x_0 \in X_0.\displaystyle\bigwedge_{i=0}^{k-1}T(\x_i, \x_{i+1}) \land \x_{k} \in X_f\\
    &=\exists \x_0 \in X_0.
    \displaystyle\bigwedge_{i=0}^{k-1}
    \left(\displaystyle\bigvee_{j \in M}P_j(\x_i) \land
        \displaystyle\bigwedge_{j \in M}P_j(\x_i) \implies \x_{i+1} =
    \Map{j}(\x_i)\right) \land \x_{k} \in X_f\\
\begin{split}
    =\exists \x_0 \in X_0.\displaystyle\bigwedge_{i=0}^{k-1} \Bigg(\displaystyle\bigvee_{j \in M}C_j\x_i\le d_j \land \displaystyle\bigwedge_{j \in M}C_j\x_i\le d_j\\
    \implies \x_{i+1} = A_j\x_i + b_j \pm \epsilon \Bigg) \land \x_{k} \in X_f
\end{split}
\end{alignat*}
\end{scriptsize}
The presence of terms $C_j\x_i$ and $A_j\x_i$ make the problem
non-linear or multi-affine when both $C_j$ (similarily $A_j$) and
$\x_i$ are simulatenously unknown. We take the route of splitting the
modeling and reachability problem into two separate problems. In the
first problem $\x_i$s are finite and known and in the second one,
$C_j$ and $A_j$ are known. We still need to connect the two problems,
which we do using a model improvement/refinement procedure.


\subsection{Search for a violation.} Given a BMC formula over affine
arithmetic, an SMT solver either returns a satisfying assignment for
it or finds it unsatisfiable. The former case provide us with an
abstract violation while the latter conveys us the absence of one in
the sound abstraction. To find the corresponding concrete violation if
it exists, a concretization step is used. If a violation is not found,
we can not conclude the absence of one. Instead, we can enrich our
data set by adding more states and re-starting. Can we do anything
else?

\subsection{Concretization} An abstract path can be concretized by
using random simulations. Same as before, we use the initial abstract
state $C_0$ in the returned abstract path and sample it to get
$\scr{N}$ concrete states ($\scr{N}$ is the concretization budget).
The states are then simulated for specified time horizon $T$ using the
given simulation function $\simulate$. If a violation is triggered, we
conclude a successful falsification and stop. If not, we refine the
abstraction.

%%%%%%%%%%%%%%%%%%%%%%%%%%%%%%

%%%%%%%%%%%%%%%%%%%%%%%%%%%%%%
\section{Refinement}
\label{sec:refine}
\todo{This section is not very clearly written}\\
A refinement of the abstraction can be carried
out by reducing the error in the PWA model. On the other hand, an
enrichment of the model be carried out by adding more behaviors
(states) in the data set.
Several ideas exist (discuss)
\begin{itemize}
    \item Add more $(\x, \x')$ pairs to the data set belonging to
        the traces obtained from $\simulate$ along the traces of
        the PWA model returned as counter-examples by the BMC.
    \item Decrease $\epsilon$ along relevant $(\x,\x')$. Such pairs
        can identified along the polytopes on the abstract
        counter-example. In an abstract path, either all polytopes can
        be refined or selectively refined by comparing the divergence
        between their predictions and observed numerical values from
        the trace produced by $\simulate$.
    \item Re-model to obtain a refined/enriched model.
    \item lather, rinse, repeat
\end{itemize}

\myipara{Note}The error is captured by $\epsilon$ for each state $\x$.
Hence, we can modify $\epsilon$ for each state individually. Or, we
can modify it for each polytope.

\myipara{Deleting states} $\x$ deemed useless for the property at
hand can also be deleted. However, if done ad-hoc, this takes away all
covergence guarentees in the presence of a robust violation in the
system $\scrS$. Perhaps we can determine conditions under which it
can be successfully applied.

\myipara{Increasing $\epsilon$} We can also increase
$\epsilon$ to apply certain parts of state space to be modeled more
coarsely than the rest. As above, an ad-hoc application will result in
loss of any convergence guarentees.

While re-modeling, it can happen that there does not exist a model
with the fixed number of polytopes $M$ and template $c$. In such a
case, we choose to increment $M := M + 1$ till we cna find a model. At
this point it is unclear how the number of constraints $c$ can be
manipulated alongside (discuss).

\section{Discussion}

If possible, we would like the time step $\tau$ to be dynamic
and dynamics based instead of fixing one for the entire system. This
can potentially decrease the depth $k$ required during BMC.

We would also like to work with the full set of LTL properties. How
can we check liveness properties? Can we extend this to finding
counter-examples for $\omega$ regular proerties?

Also, it should be noted that the above can be easily extended to
handle bounded and parameterized disturbances/inputs.

Even when a path is feasible, we need to use our previously defined
concretization step to verify if the found initial state leads to an
error state. To recall, this is done by simply using the $\simulate$
function and simulating $k$-times at and around the
$\delta$-neighbourhood of the feasible region, where $k$ and $\delta$
define the budget of the concretization step.

%%%%%%%%%%%%%%%%%%%%%%%%%%%%%%

%%%%%%%%%%%%%%%%%%%%%%%%%%%%%%
\section{Implementation}
\label{sec:impl}
\input{implementation.tex}
%%%%%%%%%%%%%%%%%%%%%%%%%%%%%%

%%%%%%%%%%%%%%%%%%%%%%%%%%%%%%
\section{Experimental Results}
\label{sec:res}
\input{results.tex}
%%%%%%%%%%%%%%%%%%%%%%%%%%%%%%

%%%%%%%%%%%%%%%%%%%%%%%%%%%%%%
\section{Conclusions}
\label{sec:concl}
\input{conclusion.tex}
%%%%%%%%%%%%%%%%%%%%%%%%%%%%%%

\bibliographystyle{abbrv}
\bibliography{references}


\end{document}

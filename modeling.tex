\section{Generating a PWA Model}

\myipara{Problem Statement.}
Given a set of $N$ states, $D_N = {\x_0,\dots,\x_n}$, error bound
$\epsilon$, find $M$ polytopes $P_{c}$, such that the below holds.

\[\forall\x\in\D_N.\exists P_j \in P.P_j(\x)\land||\Map{i}(\x) - \simulate(\tau,\x)||\le\epsilon\]

This can be stated as the below set of constraints.
\[\displaystyle\bigwedge_{\x\in D_N} \left(
        \displaystyle\bigvee_{j\in M} P_j(\x)
        \land
        \displaystyle\bigvee_{j\in M} P_j(\x)
\implies ||\Map{j}(\x) - \simulate(\tau,\x)||\le\epsilon\right)\]

In other words, we want to construct a PWA model $\hat{\scrS}$ which
predicts the next state of the system $\scrS$ within an error bounds
of $\epsilon$. Note that this only requires that for every $\x$, there
should exist atleast one polytope which predicts the next state within
the given error bounds. Hence, it is possible that other overlapping
polytopes also contain $\x$ but their asociated maps do not provide a
`good' prediction of the next state.
Such a definition might be provide for richer models with the same
number of polytopes, but it needs to be seen/proven.

The above requirement can be strengthened by searching for the PWA
model where if a polytope contains $\x$, then its associated map must
provide a `good' prediction of the next state.

\[\forall\x\in\D_N.\forall P_j \in P.P_j(\x) \implies ||\Map{i}(\x) - \simulate(\tau,\x)||\le\epsilon\]

Re-writing the above without the quantifiers, we get the below.
\[\displaystyle\bigwedge_{\x\in D_N} \left(
        \displaystyle\bigvee_{j\in M} P_j(\x)
        \land
        \displaystyle\bigwedge_{j\in M} P_j(\x)
\implies ||\Map{j}(\x) - \simulate(\tau,\x)||\le\epsilon\right)\]

As a reminder, the constraint $P_j(\x)$ can be expanded as a set of
linear constraints $C_j(\x) \le d_j$, where $C_j$ is a matrix of size
$c\times|\x|$, and $d_j$ is a vector of length $c$, where $|\x|$
represents the elements in the state vector or its dimensions.
Similarily, $\Map{j}(\x)$ represents affine map $\Map{j}(\x) =
A_j \x + b_j \pm \epsilon$, where the square matrix $A_j$ is of size
$|\x|$ and the vectors $b_j$ and $\epsilon$ are of length $|\x|$. As
the states $\x$ are pre-determined, $\simulate(\tau,\x)$ can be
computed to be a constant and the overall modeling problem is
reduced to set of affine constraints with disjunctions. Such a problem
can be solved using an SMT solver to obtain $C_j, d_j$ specifying the
$P_j$ and their associated $\Map{j}$, specified by $A_j, b_j$.

\subsection{Discussion}
It is not clear if we should also enforce the condition, that the
entire state space (region of interest) is contained in the union of
the polytopes $\displaystyle\bigcup_{j\in M} P_j \supseteq \ContStates$.
Instead, for now, we only talk about modeling all the states in the
data set.

Another important aspect of the formulation is letting the polytopes
overlap. This can be easily strenthened to uniqeness, so that every
state $\x \in D_N$ is contained by a unique $P_j$. Hence, the
resulting PWA model becomes deterministic. Although it potentially
simplify the reachability problem, it will also weaken the power of
the PWA model (more polytopes might be required to fit the same data
set).

As of now we have fixed the $\epsilon$ for the entire PWA model. This
will change when we introduce refinement.

\subsection{PWA Model as a Transition System}
The PWA model $\hat{\scrS}$ is naturally a transition system, of the
form $P_j(\x) \implies \x' := A_j \x + b_j \pm \epsilon$. Let
$T(\x,\x')$ represent the transition relation such that $\x' =
\Map{j}(\x)$.
